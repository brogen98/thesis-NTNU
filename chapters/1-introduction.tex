\chapter{Introduction}

\section{Background}
\paragraph{}When whispering to your friend in a dark room with multiple people present you want to make sure the person in front of you is actually your friend. You also want no others to hear your message. In other terms, when communicating secretly you want to know the recipient you communicate with is the intended recipient and the only recipient.\newline Equally, in tactical networks there is a need to verify that each communicating device are whom they claim to be. as well as the inherent need to not let unwanted devices be authorised and possibly commit harmful acts. Therefore, means of a secure way to authenticate is important.

\paragraph{} The purpose of this paper is to discuss existing technologies for authentication of network components, specifically end nodes, in a network without a central infrastructure. This means the infrastructure does not have a central server for authentication that end nodes may reach at any given time. The case the technologies should satisfy are technologies that may be used in a tactical or similar dynamic environment. \newline
This environment is characterised by lack of central infrastructure, varying bandwidth and the availability may differ as well. Therefore, the means of authentication has to be capable of functioning more or less standalone for each component or through interaction with a group of reachable nodes.
% insert chapter reference below
\paragraph{} The foundation for this report is a literary study of existing technologies. This study results in an analysis where each respective technology is evaluated by criteria defined as important for a tactical environment, which will be discussed further in chap. x.\newline Furthermore, there will also be extracted a subset of technologies to be analysed, which has the best means of implementation in such an environment. From there, the best of these technologies are the ones to be part of our discussion in this report. This will result in a reasoning for the best suitable technology and a design specification for how to actually implement this in a tactical environment.

\section{Problem Description}
\paragraph{} Whenever communicating with another party, you will in most cases know with whom you are speaking. In computer networks this has been a relevant concept since the inception of computer  networks, and many good solutions have since been proposed to solve the problem of authentication.

\paragraph{} The structure of an ad-hoc network in contrast to a conventional copmputer network is fundamentally quite different. 

\paragraph{} To answer this question it must bed decided on which criteria should be central when evaluating different technologies. Then a litterary study is conducted on existing technologies to gather a representative collection. From there, these technologies are analysed based on these criteria. Further, these technologies are ranked, in order to find the ones that are the most suitable. If suitable technologies are found, the best one is to be chosen in order to make a design for how to implement it in an actual tactical environment.

\section{Problem}

\section{Problem Statement}

\section{Problem Formulation}

\section{Scope and Limitations}

\section{Report Structure}
%Over the years, several thesis templates for \LaTeX{} have been developed by different groups at NTNU. Typically, there have been local templates for given study programmes, or different templates for the different study levels – bachelor, master, and PhD.\footnote{see, e.g., \url{https://github.com/COPCSE-NTNU/bachelor-thesis-NTNU} and \url{https://github.com/COPCSE-NTNU/master-theses-NTNU}}

%Based on this experience, the Community of Practice in Computer Science Education at NTNU (CoPCSE$@$NTNU)\footnote{\url{https://www.ntnu.no/wiki/display/copcse/Community+of+Practice+in+Computer+Science+Education+Home}} is hereby offering a template that should in principle be applicable for theses at all study levels. It is closely based on the standard \LaTeX{} \texttt{report} document class as well as previous thesis templates. Since the central regulations for thesis design have been relaxed – at least for some of the historical university colleges now part of NTNU – the template has been simlified and put closer to the default \LaTeX{} look and feel.

%The purpose of the present document is threefold. It should serve (i) as a description of the document class, (ii) as an example of how to use it, and (iii) as a thesis template.